%%
%% body.tex — Paper body (sections)
%%
%% This file is \input{} by main.tex.
%% Write all your sections here; do not edit main.tex for content.
%%
%% CITATION COMMANDS
%%   \legal{key}           → footnote: full citation [Short].   (first use)
%%   \legal[pinpoint]{key} → footnote: full citation at … [Short].
%%   \legalname{key}       → italicised case/act name in text (no footnote)
%%   \legalshort{key}      → short form in text (no footnote)
%%
%% PINPOINT SYNTAX (McGill Guide §1.3)
%%   Cases     → para 32 | paras 27-31 | paras 18, 23
%%   Books     → p 42    | pp 148-155  | pp 214, 218
%%   Articles  → p 5     | pp 5-7
%%

%% ────────────────────────────────────────────────────────────────────────────
\section{Introduction}
%% ────────────────────────────────────────────────────────────────────────────

Recent Supreme Court of Canada (SCC) copyright cases of particular relevance to
libraries include \legalname{york}\legal{york} and
\legalname{socan}.\legal{socan}
In \legalshort{york}, the SCC rejected Access Copyright's attempt to impose
mandatory tariffs upon York University, but the SCC did not address York
University's claim of fair dealing.
In \legalshort{socan}, the SCC supported technological neutrality in ruling that
``[s]imilar to offline distribution, downloading or streaming works will
continue to engage only one copyright interest and require paying one
royalty.''\legal[para 112]{socan}

Beyond copyright, \legalshort{socan} is also noteworthy for adding a category
for correctness review to those recognised by the SCC in
\legalname{vavilov}\legal{vavilov} as reasons to ``derogat[e] from the
presumption of reasonableness review.''\legal[para 69]{vavilov}
As the SCC noted in \legalshort{socan}, \legalshort{vavilov} allowed that there
might be additional ``exceptional'' circumstances beyond those recognised in
\legalshort{vavilov} that require this derogation.\legal[para 27]{socan}
The additional category added in \legalshort{socan} is ``when courts and
administrative bodies have concurrent first instance jurisdiction over a legal
issue in a statute.''\legal[para 28]{socan}

%% ────────────────────────────────────────────────────────────────────────────
\section{Analysis}
%% ────────────────────────────────────────────────────────────────────────────

The \legalname{copyrightact} has been interpreted broadly by Canadian
courts.\legal{copyrightact}
In \legalname{cch}, the SCC adopted a large and liberal interpretation of
``fair dealing''.\legal[para 48]{cch}
For commentary on that decision, see Scassa\legal{scassa} and de Beer.\legal{debeer}

The work of Ziff remains an important resource on property law.\legal{ziff}
The author discusses the numerus clausus principle at length.\legal[pp 148-155]{ziff}

For a detailed treatment see also Judge and Gervais.\legal[pp 5-7]{judge}
Gervais has written a complementary structural account.\legal{gervais}

The \legalshort{cch} decision remains authoritative on fair dealing.\legal[para 48]{cch}

%% ────────────────────────────────────────────────────────────────────────────
\section{Conclusion}
%% ────────────────────────────────────────────────────────────────────────────

The cases surveyed above confirm that \legalname{socan} and \legalname{vavilov}
together represent a significant development in administrative and copyright
law.\legal[paras 27-31]{socan}
